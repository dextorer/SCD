%----------------------------------------------------------------------------------------
%	IMPLEMENTAZIONE
%----------------------------------------------------------------------------------------

\section*{Implementazione}
\addcontentsline{toc}{section}{\protect\numberline{}Implementazione}%
\label{sec:implementazione}

Paragrafo introduttivo.

\subsection*{Concorrenza}
\addcontentsline{toc}{subsection}{\protect\numberline{}Concorrenza}%
\label{sec:implemetazione_concorrenza}

Concorrenza.

\subsection*{Distribuzione}
\addcontentsline{toc}{subsection}{\protect\numberline{}Distribuzione}%
\label{sec:implementazione_distribuzione}

In questa sezione viene discussa l'implementazione delle componenti distribuite del software, ovvero \emph{Field} e \emph{Manager}.

La comunicazione tra la componente \emph{Core} e le componenti \emph{Field} e \emph{Manager} e viceversa avviene facendo agire \emph{Core} come un web server. Ci\`{o} viene fatto grazie all'utilizzo di AWS (Ada Web Server). All'avvio del sistema, viene inizializzato un web server all'indirizzo \emph{localhost} alla porta 28000 a cui \emph{Field} e \emph{Manager} si connettono e con il quale successivamente scambiano informazioni.\\\\ 
La comunicazione tra le componenti distribuite \emph{Field} e \emph{Manager} e la componente centrale \emph{Core} \`{e} stata implementata con un modello di comunicazione tipico di un’architettura client-server. Ad esempio, ciascuna istanza di \emph{Manager}, una volta avviata, invia una richiesta HTTP GET a \emph{Core} richiedendo le statistiche dei giocatori della squadra. La componente \emph{Field} effettua invece chiamate  HTTP GET verso \emph{Core} per richiedere azioni quali l'avvio di una nuova partita, l'avvio del secondo tempo della partita corrente, la messa in pausa della partita corrente, ma anche la terminazione forzata della partita.

Per permettere tale tipo di comunicazione a partire dalle componenti distribuite scritte in Java \`{e} stata utilizzata la libreria open source Apache HttpComponents, la quale fornisce una completa implementazione del protocollo HTTP in Java e consente di avere accesso a maggiori funzionalit\`{a} e flessibilit\`{a} rispetto al package standard \emph{java.net} di Java.

In generale, le richieste provenienti da \emph{Field} e \emph{Manager} sono inizialmente ricevute dal modulo \emph{Soccer.Server.Callbacks} che si occupa di verificare di quale tipo di richiesta si tratta (i metodi a disposizione di delle componenti distribuite sono consultabili in sezione ~\ref{sec:analisi_distribuzione_bridge_input}) per poi inoltrarla correttamente verso la componente \emph{Core} tramite il modulo bridge input. Nel caso in cui il client richiedente attenda una risposta da \emph{Core} (e.g. il client ha chiesto le statistiche dei giocatori) il modulo \emph{Soccer.Server.Callbacks} si occuper\`{a} di fornirla al client.\\\\
La comunicazione di \emph{Core} con le componenti distribuite \emph{Field} e \emph{Manager} \`{e} invece vista come un modello di comunicazione di tipo publisher-subscriber. Tale server mette infatti a disposizione i seguenti canali di comunicazione (ovvero apre dei websocket) ai quali le componenti interessate si iscrivono per ricevere informazioni:
\begin{itemize}
\item \emph{/managerVisitors/registerForStatistics} \`{e} il websocket su cui una delle istanze di \emph{Manager} rimane in ascolto, in particolare l'istanza che rappresenta la squadra che gioca ``fuori casa''
\item \emph{/managerHome/registerForStatistics} \`{e} il websocket su cui una delle istanze di \emph{Manager} rimane in ascolto, in particolare l'istanza che rappresenta la squadra che gioca ``in casa''
\item \emph{/field/registerForEvents} \`{e} il websocket a cui si connette \emph{Field}
\end{itemize}
\noindent I websocket vengono utilizzati da \emph{Core} per mandare eventi ed informazioni alle altre componenti. Le istanze di \emph{Manager} riceveranno informazioni riguardanti le statistiche dei giocatori e la formazione della squadra, mentre \emph{Field} riceve tutti gli eventi correlati con la partita in corso. 

Tramite l'utilizzo dei websocket viene mantenuto attivo un canale di comunicazione tra \emph{Core} e le componenti distribuite, permettendo cos\`{i} la fruizione di contenuti verso \emph{Field} e \emph{Manager}. Il principale vantaggio nell'utilizzo dei websocket  \`{e} sta nel fatto che l'invio di nuovi contenuti disponibili da parte dell'entit \`{a} produttore verso le entit \`{a} consumatori avviene senza alcuna richiesta o sollecitazione da parte dei consumatori: non appena  \`{e}  disponibile un nuovo contenuto, il produttore lo invia autonomamente ai consumatori, i quali sono in ascolto sui corrispondenti websocket aperti.

Ada Web Server mette a disposizione una serie di API per facilitare l'uso dei websocket. Per aprire un nuovo websocket \`{e} sufficiente utilizzare la funzione \emph{Register} passando come parametro l'indirizzo su cui renderlo disponibile. Per inviare dei dati vi \`{e} la funzione \emph{Send}, alla quale \`{e} possibile specificare il websocket sul quale spedire i dati e anche il tipo di dati spedito.\\\\
Come spiegato in sezione ~\ref{sec:analisi_distribuzione_bridge_output}, non vi \`{e} un continuo stream di informazioni verso le componenti distribuite, ma viene utilizzato un buffer all'interno di bridge output il cui scopo \`{e} bilanciare l'invio di dati: se da un lato un throughtput troppo basso comprometterebbe la rappresentazione grafica della partita, dall'altro un invio troppo frequente di aggiornamenti potrebbe causare una congestione di rete.

\subsubsection{Codifica delle Informazioni}
Tutti i messaggi scambiati secondo i modelli appena descritti sono codificati utilizzando il formato JSON, il quale \`{e} un formato di testo completamente indipendente dal linguaggio di programmazione, ma utilizza convenzioni conosciute dai programmatori di linguaggi della famiglia del C, come C, C++, C\#, Java, JavaScript, Perl, Python, e molti altri. Questa caratteristica fa di JSON un linguaggio ideale per lo scambio di dati. 

La componente \emph{Core} processa i dati ricevuti in formato JSON utilizzando\emph{GNATColl}, o GNAT Component Collection, una libreria che mette a disposizione degli ADA package general purpose aggiuntivi. Tra di essi vi \`{e} il package \emph{GNATColl.JSON}, il quale permette sia la creazione di oggetti JSON che il parsing di tale tipo di dati ricevuti da \emph{Core}.

Per gestire le informazioni in formato JSON le componenti distribuite, \emph{Manager} e \emph{Field}, utilizzano \emph{Gson}, una libreria open source  inizialmente sviluppata da Google. Questa libreria \`{e} stata scelta per la semplicit\`{a} d'uso e la versatilit\`{a} delle API messe a disposizione. La conversione di un oggetto Java nella sua rappresentazione in JSON \`{e} effettuata utilizzando il methodo \emph{toJson()} messo a disposizione dalla libreria. Similmente, la conversiona di una semplice stringa JSON nel corrispondente oggetto Java \`{e} possibile utilizzando il metodo \emph{fromJson()}. 

\subsubsection{Interfacce Grafiche}
Le GUI delle componenti \emph{Field} e \emph{Manager} sono state realizzate utilizzando il linguaggio Java. Questa scelta \`{e} stata guidata dal fatto che Java \`{e} un linguaggio che garantisce un certo livello di portabilit\`{a} del sistema. In particolare, la grafica ed il layout sono stati implementati utilizzando il framework Swing di Java.