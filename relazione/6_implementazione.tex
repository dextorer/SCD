%----------------------------------------------------------------------------------------
%	IMPLEMENTAZIONE
%----------------------------------------------------------------------------------------

\section*{Implementazione}
\addcontentsline{toc}{section}{\protect\numberline{}Implementazione}%
\label{sec:implementazione}

Paragrafo introduttivo.

\subsection*{Concorrenza}
\addcontentsline{toc}{subsection}{\protect\numberline{}Concorrenza}%
\label{sec:implemetazione_concorrenza}

Concorrenza.

\subsection*{Distribuzione}
\addcontentsline{toc}{subsection}{\protect\numberline{}Distribuzione}%
\label{sec:implementazione_distribuzione}

In questa sezione viene discussa l'implementazione delle componenti distribuite del software, ovvero \emph{Field} e \emph{Manager}.

La comunicazione tra la componente \emph{Core} e le componenti \emph{Field} e \emph{Manager} e viceversa avviene facendo agire \emph{Core} come un web server. Ci\`{o} viene fatto grazie all'utilizzo di AWS (Ada Web Server). All'avvio del sistema, viene inizializzato un web server all'indirizzo \emph{localhost} alla porta 28000 a cui \emph{Field} e \emph{Manager} si connettono e con il quale successivamente scambiano informazioni.\\\\ 
La comunicazione tra le componenti distribuite \emph{Field} e \emph{Manager} e la componente centrale \emph{Core} \`{e} stata implementata con un modello di comunicazione tipico di un’architettura client-server. Le richieste provenienti da \emph{Field} e \emph{Manager} sono inizialmente ricevute dal modulo \emph{Soccer.Server.Callbacks} che si occupa di verificare di quale tipo di richiesta si tratta (i metodi a disposizione di delle componenti distribuite sono consultabili in sezione ~\ref{sec:analisi_distribuzione_bridge_input}) per poi inoltrarla correttamente verso la componente \emph{Core} tramite il modulo bridge input. Nel caso in cui il client richiedente attenda una risposta da \emph{Core} (e.g. il client ha chiesto le statistiche dei giocatori) il modulo \emph{Soccer.Server.Callbacks} si occuper\`{a} di fornirla al client.\\\\
La comunicazione di \emph{Core} con le componenti distribuite \emph{Field} e \emph{Manager} \`{e} invece vista come un modello di comunicazione di tipo publisher-subscriber. Tale server mette infatti a disposizione i seguenti canali di comunicazione (ovvero apre dei socket) ai quali le componenti interessate si iscrivono per ricevere informazioni:
\begin{itemize}
\item \emph{/managerVisitors/registerForStatistics} \`{e} il socket su cui una delle istanze di \emph{Manager} rimane in ascolto, in particolare l'istanza che rappresenta la squadra che gioca ``fuori casa''
\item \emph{/managerHome/registerForStatistics} \`{e} il socket su cui una delle istanze di \emph{Manager} rimane in ascolto, in particolare l'istanza che rappresenta la squadra che gioca ``in casa''
\item \emph{/field/registerForEvents} \`{e} il socket a cui si connette \emph{Field}
\end{itemize}

\noindent I socket vengono utilizzati da \emph{Core} per mandare eventi ed informazioni alle altre componenti. Le istanze di \emph{Manager} riceveranno informazioni riguardanti le statistiche dei giocatori e la formazione della squadra, mentre \emph{Field} riceve tutti gli eventi correlati con la partita in corso. 

Come spiegato in sezione ~\ref{sec:analisi_distribuzione_bridge_output}, non vi \`{e} un continuo stream di informazioni verso le componenti distribuite, ma viene utilizzato un buffer all'interno di bridge output il cui scopo \`{e} bilanciare l'invio di dati: se da un lato un throughtput troppo basso comprometterebbe la rappresentazione grafica della partita, dall'altro un invio troppo frequente di aggiornamenti potrebbe causare una congestione di rete.

\subsubsection{Codifica delle Informazioni}
Tutti i messaggi scambiati secondo i modelli appena descritti sono codificati utilizzando il formato JSON, il quale \`{e} un formato di testo completamente indipendente dal linguaggio di programmazione, ma utilizza convenzioni conosciute dai programmatori di linguaggi della famiglia del C, come C, C++, C\#, Java, JavaScript, Perl, Python, e molti altri. Questa caratteristica fa di JSON un linguaggio ideale per lo scambio di dati.

\subsubsection{Interfacce Grafiche}
Le GUI delle componenti \emph{Field} e \emph{Manager} sono state realizzate utilizzando il linguaggio Java. Questa scelta \`{e} stata guidata dal fatto che Java \`{e} un linguaggio che garantisce un certo livello di portabilit\`{a} del sistema. In particolare, la grafica ed il layout sono stati implementati utilizzando il framework Swing di Java.