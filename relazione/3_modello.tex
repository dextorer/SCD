%----------------------------------------------------------------------------------------
%	MODELLO
%----------------------------------------------------------------------------------------

\section*{Il modello}
\addcontentsline{toc}{section}{\protect\numberline{}Il modello}%
\label{sec:modello}

Nel Capitolo~\ref{sec:entita_coinvolte} sono state individuate ed analizzate le entita' che caratterizzano il sistema. La fase successiva, su cui questo capitolo si focalizza, consiste nel definire le interazioni che legano le suddette entita', in maniera tale da poter derivare un modello per il sistema.

\subsection*{Interazione tra giocatori: il controllore}
\addcontentsline{toc}{subsection}{\protect\numberline{}Interazione tra giocatori: il controllore}%
\label{sec:modello_interazione_giocatori}

%Interazione tra giocatori: il controllore.
I giocatori sono i principali attori del sistema. Un giocatore si sposta dentro e fuori dal campo, e' in grado di compiere diversi tipi di azioni (Sezione~\ref{sec:entita_giocatori}), che decide sia sulla base del proprio stato che dello stato della propria squadra e della partita. Un aspetto fondamentale vede i giocatori, cosi' come avviene nel mondo reale, operare in maniera parallela e concorrente tra di loro: questo implica che i risultati delle loro azioni debbano riflettersi sullo stato di gioco, che come descritto in Sezione~\ref{sec:entita_stato}, detiene tutte le informazioni che definiscono la partita in un dato istante. Si pone pero' un problema: i giocatori, per poter conoscere le informazioni che permetteranno loro di decidere la prossima azione da compiere, i giocatori devono accedere allo stato, che tuttavia e' unico. E' dunque necessario regolamentare l'accesso a tale risorsa, in maniera tale da preservarne la consistenza e la correttezza.\\

L'organo che si occupa di permettere l'interazione tra diversi giocatori e' il \textit{controllore}. In quanto entita' centrale di controllo, i suoi molteplici compiti possono essere raggruppati come segue:

\begin{enumerate}
	\item Permettere ai giocatori di accedere allo stato, sia per leggerne le informazioni sia per poterle modificare
	\item Sequenzializzare gli accessi allo stato da parte dei giocatori, al fine di non creare inconsistenze sulle informazioni in esso contenute
	\item Ricoprire il ruolo di arbitro di gioco ``onnisciente'' (verra' dettagliato in Sezione~\ref{sec:modello_verifica_arbitro})
\end{enumerate}

Come gia' accennato precedentemente, un giocatore decide la sua prossima mossa sulla base dello stato corrente della partita: ad esempio, trovandosi in possesso della palla nell'area avversaria e senza nessun giocatore a marcarlo, e' ragionevole che il giocatore decida di tirare per provare a realizzare un goal. Al tempo stesso, l'azione che egli compie si ripercuote sullo stato della partita, andandone a modificare una parte: di nuovo, se il giocatore segna, il punteggio della partita cambia e i giocatori ritornano in posizione di partenza per ricominciare il gioco. Questo meccanismo mette in luce una necessita' tale per cui le operazioni che alterano lo stato siano quanto piu' possibile sequenziali ed atomiche. Sebbene questo argomento verra' trattato ampiamente nel Capitolo~\ref{sec:analisi_architetturale}, e' importante capire il motivo dell'importanza di queste due caratteristiche.\\

L'assunzione di atomicita' si rivela particolamente critica quando si assume di operare in condizioni di prerilascio dei processi. Sotto questa condizione, un processo correntemente in esecuzione puo' essere ``temporaneamente fermato'' (prerilasciato) in favore di un altro processo, che entra quindi in esecuzione al suo posto; una simile condizione si verifica, ad esempio, se il processo corrente ha una priorita' inferiore di quello che vuole subentrare. Per spiegare come il prerilascio minacci la consistenza dello stato della partita, si consideri una situazione dove due giocatori avversari si contendono il possesso della palla, che giace inerte in mezzo a loro. Entrambi i giocatori leggono lo stato e avviano il processo di decisione della prossima azione. Tuttavia, il processo relativo al primo giocatore viene prerilasciato. Nel frattempo, il secondo giocatore conquista la palla e si muove verso la porta avversaria. Quando il primo giocatore torna in esecuzione si trova in uno stato inconsistente, in quanto lo stato e' cambiato senza che lui lo sappia. Ancora peggio, se il primo giocatore procede con la scrittura della sua azione, lo stato verra' modificato erroneamente, potenzialmente togliendo il possesso di palla al secondo giocatore.\\

[TODO] aggiungere la parte di mutua esclusione

\subsection*{Verifica sullo stato di gioco: l'arbitro}
\addcontentsline{toc}{subsection}{\protect\numberline{}Verifica sullo stato di gioco: l'arbitro}%
\label{sec:modello_verifica_arbitro}

Affiche' una partita si svolga secondo le regole e le modalita' stabilite dal gioco del calcio c'e' bisogno di un arbitro che regoli l'andamento del gioco. Le mansioni dell'arbitro sono molteplici:

\begin{itemize}
	\item Sancire l'inizio e la fine dei tempi di gioco (inizio primo tempo - fine primo tempo - inizio secondo tempo - fine partita)
	\item Fermare il gioco e farlo riprendere a seguito (e.g. una rimessa laterale)
	\item Segnalare eventuali irregolarita' da parte dei giocatori (e.g. un fallo)
	\item Tenere il conto dei gol segnati da entrambe le squadre, cosi' da decretare il vincitore alla fine della partita
\end{itemize}

L'arbitro deve quindi essere in grado di controllare tutte le mosse dei giocatori, cosi' come lo stato e la posizione della palla e la durata della partita fino a quel momento. Nella realta', il ruolo dell'arbitro e' assegnato ad un essere umano, che quindi non e' infallibile: si pensi ad esempio ad un fallo che viene commesso irregolarmente alle sue spalle mentre lui e' impegnato ad assegnare un calcio d'angolo. In questa simulazione si assume piu' semplicemente che l'arbitro sia ``onnisciente'', ovvero abbia la facolta' di analizzare ogni singola mossa di ciascun giocatore e della palla, in maniera da poter segnalare immediatamente ogni irregolarita' oppure fermare il gioco all'occorrenza.\\

Si ha cosi' che il controllore, descritto nel paragrafo precedente, ricopre anche il ruolo di arbitro. Questa decisione ha delle ripercussioni non solo nello svolgersi del gioco (l'arbitro e' onnisciente), ma anche nell'assegnazione delle risorse di calcolo. Infatti, se l'arbitro fosse soggetto agli stessi vincoli di esecuzione dei giocatori, andrebbe a concorrere assieme a loro per l'esecuzione sulla CPU come task a se stante; questa situazione non si verifica invece nel caso in cui sia il controllore ad essere anche arbitro, essendo l'entita' centrale che si occupa di eseguire a tutti gli effetti le mosse dei giocatori. Maggiori dettagli sull'implementazione dell'arbitro verranno esposti in Sezione~\ref{sec:implemetazione_concorrenza}.

[TODO] aggiungere il fatto che l'arbitro deve gestire anche gli eventi che provengono dalla distribuzione e in che ordine vengono processati gli eventi (vedere documento SCDNOTES)

\subsection*{La palla in movimento: l'entita' e l'agente di movimento}
\addcontentsline{toc}{subsection}{\protect\numberline{}La palla in movimento: l'entita' e l'agente di movimento}%
\label{sec:modello_palla_agente_movimento}

%La palla in movimento: l'entita' e l'agente di movimento.

La palla e' una parte fondamentale del modello della simulazione, in quanto il suo possesso viene conteso dai giocatori che devono tirarla in porta, segnando un gol per la loro squadra. Il suo comportamento puo' essere definito come una macchina a stati, schematizzato nel seguente diagramma:\\

[TODO] aggiungere diagramma macchina a stati della palla

In ogni momento della partita, la palla occupa una delle celle del campo. Nel caso sia posseduta da un giocatore, essi condividono la stessa cella; in caso contrario, la palla occupa la cella in cui si trova. Inoltre, la palla puo' trovarsi solamente in due stati: inerzia e moto. Una palla in movimento si puo' avere quando il giocatore che la controlla si sposta con essa; inoltre, si ha una palla in movimento anche quando un giocatore la passa verso un altro giocatore oppure effettua un tiro verso la porta avversaria. Al contrario, una palla e' inerte se non e' controllata da nessun giocatore, solitamente quando un passaggio o un tiro mancano il bersaglio (e.g. un passaggio troppo debole).\\

Ad ogni modo, la palla e' un'entita' passiva, che non compie azioni proprie ma che subisce azioni di altre entita' attive (i giocatori). Di conseguenza, quando un giocatore effettua un passaggio oppure un tiro, la palla deve essere spostata da un'entita' che pero' non puo' essere il giocatore stesso: in altre parole, si tratta di simulare l'impressione di un moto alla palla a seguito di un'azione del giocatore che la controlla. Questa mansione e' ricoperta dal cosiddetto \textit{agente di movimento}. L'agente di movimento e' un'entita' concorrente agli altri giocatori, il cui unico compito e' quello di spostare la palla in una determinata direzione fino a che la potenza ad essa impressa e' sufficiente a farla avanzare alla cella successiva. Una volta completato il suo compito, smette di eseguire in attesa del prossimo spostamento da effettuare. In alcuni casi, l'agente di movimento viene volutamente bloccato attraverso l'arbitro, ad esempio nel caso in cui la palla esca dal campo e sia necessario assegnare una rimessa oppure un calcio d'angolo.

\subsection*{Modifiche sulle squadre: gli allenatori}
\addcontentsline{toc}{subsection}{\protect\numberline{}Modifiche sulle squadre: gli allenatori}%
\label{sec:modello_squadre_allenatori}

Modifiche sulle squadre: gli allenatori.

\subsection*{Gestione delle fasi di gioco: entita' di gioco}
\addcontentsline{toc}{subsection}{\protect\numberline{}Gestione delle fasi di gioco: entita' di gioco}%
\label{sec:modello_fasi_game_entity}

Gestione delle fasi di gioco: entita' di gioco.
