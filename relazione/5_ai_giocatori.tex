%----------------------------------------------------------------------------------------
%	INTELLIGENZA ARTIFICIALE DEI GIOCATORI
%----------------------------------------------------------------------------------------

\section{IA dei giocatori}
\label{sec:ai_giocatori}

In questa sezione viene descritta la struttura dell'intelligenza artificiale dei giocatori.

% Programmazione Logica: Prolog
\subsection{Programmazione Logica: Prolog}
Per creare il sistema decisionale dei giocatori utilizzeremo Prolog, un linguaggio di programmazione che trova le sue radici nella logica del primo ordine.
\footnote{Durante lo sviluppo del progetto \`{e} stato utilizzato SWI-Prolog, un'implementazione open source di Prolog compatibile con ogni piattaforma e che mette a disposizione un ampio set di tools per lo sviluppo. Sito ufficiale: \href{http://www.swi-prolog.org/}{http://www.swi-prolog.org}.}.
Essendo un linguaggio di programmazione ispirato alla logica del primo ordine, un programma scritto in Prolog \`{e} un insieme di predicati formati da una concatenazione di \emph{precondizioni}, ovvero dei fatti che devono essere tutti veri affinch\`{e} la \emph{testa} della clausola sia vera.

Prolog risolve le clausole utilizzando l'inferenza logica in maniera molto efficiente, ma senza nessun controllo su cicli o cammini infiniti. Ci\`{o} lo rende molto veloce se gli viene fornito un corretto insieme di clausole, ma incompleto altrimenti. Infatti la politica adottata da Prolog \`{e} quella di ``scaricare'' sul programmatore la responsabilit\`{a} di scrivere programmi corretti ed efficienti.\\

\subsection{Struttura dell'IA}
\label{sec:struct_ia}
In questa sezione viene analizzata la struttura dell'intelligenza artificale creata in Prolog.

Il sistema \`{e} diviso in due parti principali ciascuna delle quali \`{e} composta da una serie di predicati logici scritti in Prolog. La prima parte \`{e} formata da predicati che descrivono quello che il giocatore sa riguardo a ci\`{o} che lo circonda, detta anche \emph{base di conoscenza}. Per costruirsi tali predicati ciascun giocatore deve procurarsi informazioni riguardanti la situazione in prossimit\`{a} della sua posizione (la sua posizione, quale squadra ha la palla, quali giocatori sono vicini a lui e cos\`{i} via) interrogando lo stato.

Pi\`{u} precisamente, per ciascuna informazione che il giocatore chiede ed ottiene dallo stato, viene costruito il corrispondente predicato logico scritto in Prolog. Vi sar\`{a} quindi, ad esempio, un predicato per la posizione del giocatore, un predicato per la posizione della palla, un predicato per ciascun giocatore presente nelle vicinanze del giocatore corrente e cos\`{i} via. Ovviamente nello stato sono presenti tipi di dato non interpretabili da Prolog. \`{E} quindi necessario un passaggio intermedio che ci permetta di trasformare le informazioni contenute nello stato in un formato comprensibile a Prolog. \\
Questa conversione \`{e} molto semplice ed avviene man mano che il giocatore riceve dati dallo stato. Prendiamo come esempio la posizione attuale del giocatore. Il giocatore richiede allo stato questa informazione, il quale gli restituisce un oggetto di tipo \emph{Coordinate}, un tipo di dato che rappresenta la posizione con una coppia di interi (x,y). A questo punto avviene la conversione. Dall'oggetto di tipo Coordinate sono estratti i due interi x e y, i quali vengono inseriti in una nuova stringa Ada contenente la posizione del giocatore, scritta sotto forma di predicato logico capibile da Prolog. Se supponiamo che la posizione del giocatore presente nello stato sia la coppia di interi (10,15), la corrispondente stringa generata sarebbe: 
\begin{equation*}
	\texttt{ position(10, 15).} 
\end{equation*}
\noindent Questa operazione viene ripetuta per tutte le richieste che il giocatore fa allo stato al fine di ottenere un quadro generale di ci\`{o} che sta succedendo nella partita.\\
Una volta esaurite tutte le richieste allo stato, le varie stringhe Ada cos\`{i} ottenute vengono concatenate in un'unica nuova stringa, che rappresenterà la base di conoscenza del giocatore. Questa nuova stringa viene passata come parametro, tramite uno script adibito all'esecuzione del motore di Prolog e alla ricezione del risultato dell'inferenza logica effettuata sulla base di conoscenza fornita come parametro, alla seconda parte del sistema.

Questa seconda parte \`{e} formata da predicati che rappresentano le azioni che i giocatori possono effettuare. Questa parte \`{e} a sua volta suddivisa in tre categorie principali:
\begin{itemize}
\item \emph{Actions}: contiene tutte le clausole riguardanti le azioni che un giocatore `normale' pu\`{o} effettuare. Le categorie di azioni a disposizione di un giocatore sono passaggio (\emph{pass}), tiro (\emph{shot}), movimento (\emph{move}), contrasto (\emph{tackle}) e `prendi la palla' (\emph{catch}); 
\item \emph{Keeper}: comprende le azioni che il giocatore pu\`{o} fare se assume il ruolo di portiere. Il motivo dell'introduzione di questa distinzione tra giocatore `normale' e portiere \`{e}  che il primo non solo ha a disposizione pi\`{u}  azioni possibili rispetto al secondo, ma ha anche un comportamento differente. Ci\`{o}  non dovrebbe sorprendere visto che durante la partita il portiere sta la maggior parte del tempo fermo nella sua porta, mentre un giocatore normale si sposta nel campo ed interagisce con altri giocatori molto pi\`{u}  spesso. Inoltre alcuni eventi di gioco non interessano minimamente il portiere, come ad esempio una rimessa laterale o un semplice calcio di punizione eseguito vicino alla met\`{a}  campo, mentre il giocatore normale potrebbe dover spostarsi in una posizione particolare a causa di essi. Le azioni contenute in questo file sono quindi in numero minore e, in alcuni casi, eseguite diversamente. Per tutti questi motivi \`{e}  stato ritenuto opportuno utilizzare un set di azioni `personalizzato' solo per il portiere, nel qual caso durante l'esecuzione del programma i predicati in `Keeper' verranno utilizzati al posto dei predicati in `Actions';
\item \emph{Utilities}: contiene clausole ausiliarie utilizzate dai predicati in `Actions' e `Keeper'. Alcuni esempi di clausole ausiliarie sono l'aggiunta di un elemento ad una lista, il calcolo della distanza tra due punti e il calcolo della corretta coppia di coordinate in cui spostarsi. Queste clausole sono state inserite in un modulo distinto per facilitare la comprensione del programma finale.
\end{itemize}
\noindent Le azioni che possono essere scelte dal giocatore in un dato momento sono determinate dallo stato della partita, dalla presenza di eventi specifici (rimessa, punizione e così via) e dalla situazione specifica nelle vicinanze del giocatore. Pi\`{u} precisamente, l'azione da compiere in un dato istante \`{e} il risultato di un'inferenza logica effettuata sulle clausole incluse nella base di conoscenza del giocatore, ovvero sull'input ricevuto dalla prima parte del sistema.

\subsection{Workflow}
Il processo decisionale che porta il giocatore ad eseguire un'azione \`{e} quindi il seguente:
\begin{enumerate}
 \item Il giocatore interroga lo stato per ottenere informazioni riguardanti le sue immediate vicinanze e lo stato complessivo della partita
 \item L'algoritmo di conversione delle informazioni da tipo di dato Ada a stringhe contenenti i predicati scritti secondo i formalismi di Prolog elabora i dati ottenuti dallo stato
 \item Viene lanciato, tramite un apposito script, il motore Prolog fornendo come input la concatenazione di predicati ottenuta al passo precedente
 \item Prolog esegue un inferenza logica sull'input ricevuto al passo precedente e fornisce in ouput la mossa ottimale per il giocatore data la situazione corrente
 \end{enumerate} 
\noindent La mossa ottenuta in output \`{e} ricevuta sotto forma di stringa Ada contenente un predicato Prolog costituito dalla mossa da effettuare e una coppia di interi (x,y) che rappresentano la coordinata sulla quale il giocatore andr\`{a} ad effettuare la mossa. Questa stringa viene scomposta in tre sottostringhe: una contenente la mossa da effettuare (ad esempio \emph{shot}, il che sta ad indicare che il giocatore ``ha deciso'' di tirare), una stringa per la componente x della coordinata e infine una per la componente y. A partire da queste sottostringhe viene creata la mossa che il giocatore cercher\`{a} di scrivere sullo stato, avendo cura di creare un oggetto di tipo \emph{Coordinate} a partire dalle due sottostringhe contenenti le componenti della coordinata finale.