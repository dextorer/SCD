%----------------------------------------------------------------------------------------
%	INTRODUZIONE
%----------------------------------------------------------------------------------------

\section{Introduzione}
\label{sec:introduzione}

Il seguente progetto e' il frutto del lavoro svolto nell'ambito del corso di \emph{Sistemi Concorrenti e Distribuiti}. Il progetto proposto \`{e} stato tratto da un concorso annuale chiamato \emph{The Ada Way} ed organizzato da Ada-Europe. In particolare, viene fatto riferimento all'edizione del concorso 2010/2011, anno in cui il tema proposto \`{e} stato lo sviluppo di un simulatore di una partita di calcio in cui gli attori della partita, ovvero giocatori e arbitri, sono interamente controllati dal computer e devono agire in maniera indipendente e al tempo stesso concorrente. L'utente ha la possibilit\`{a} di interagire con la partita in corso, assumendo il ruolo di allenatore delle squadre.\\

Questo documento ha lo scopo di descrivere l'approccio utilizzato per risolvere il problema e si concentra prevalentemente sulle problematiche di concorrenza e distribuzione incontrate durante lo sviluppo del software.

\subsection{Scopo del progetto}
\label{sec:scopo_del_progetto}

Lo scopo del progetto consiste nel realizzare un software che simuli lo svolgimento di una partita di calcio secondo il relativo regolamento. Il software \`{e} presenta una componente principale, nella quale avvengono le dinamiche di concorrenza che vedranno le entit\`{a} in gioco interagire tra di loro, ed alcune componenti di distribuzione, che interagiscono con la componente principale e che vengono pilotate dall'utente.

\subsection{Struttura del documento}
\label{sec:struttura_del_documento}

Il documento assume la seguente struttura:

\begin{itemize}
	\item il Capitolo 1 consiste in una breve introduzione al progetto e al suo contesto
	\item nel Capitolo 2 viene proposta l'analisi delle problema e la successiva definizione di un modello per la soluzione
	\item nel Capitolo 3 si effettua un'analisi architetturale approfondita della soluzione, volta ad evidenziare le scelte relative agli aspetti di concorrenza e distribuzione
	\item il Capitolo 4 introduce brevemente l'intelligenza artificiale dei giocatori
	\item nel Capitolo 5 vengono descritte le scelte implementative adottate
	\item il Capitolo 6 serve come manuale utente e dettaglia le informazioni relative alla configurazione e all'esecuzione del software
	\item nel Capitolo 7 sono presentate le conclusioni
\end{itemize}
