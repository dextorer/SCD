%----------------------------------------------------------------------------------------
%	CONCLUSIONI
%----------------------------------------------------------------------------------------

\section*{Conclusioni}
\addcontentsline{toc}{section}{\protect\numberline{}Conclusioni}%
\label{sec:conclusioni}

Il progetto ha avuto come obiettivo quello di creare una simulazione di una partita di calcio con componenti di concorrenza e di distribuzione.\\

Per quanto concerne la concorrenza abbiamo appreso come sia importante creare in fase di desigh un'architettura che sia a prova di deadlock e di starvation, e che allo stesso permetta un certo livello di controllo all'interno del sistema. La scelta di avere un unico punto di sincronizzazione ha ridotto il potenziale parallelismo tra le entita' aumentando il livello di contesa nell'ottenere le risorse. In fase di sviluppo e' stato complicato trovare gli errori quando le dinamiche non rispettavano le attese, il modello di concorrenza di Ada ha permesso di utilizzare meccanismi che rispecchiassero i comportamenti che volevamo ottenere.\\

Distribuzione, che pacco.\\

IA in Prolog, croce e delizia, genio e sregolatezza.