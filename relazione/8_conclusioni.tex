%----------------------------------------------------------------------------------------
%	CONCLUSIONI
%----------------------------------------------------------------------------------------

\section{Conclusioni}
\label{sec:conclusioni}

Il progetto ha avuto come obiettivo quello di creare una simulazione di una partita di calcio con componenti di concorrenza e di distribuzione.\\

Per quanto concerne la concorrenza, abbiamo appreso l'importanza di creare in fase di design un'architettura che sia esente dalle condizioni di deadlock e di starvation, ma che permetta allo tempo stesso un certo livello di controllo all'interno del sistema. La scelta di avere un unico punto di sincronizzazione, identificato con lo stato, ha ridotto il potenziale parallelismo tra le entità, aumentando il livello di contesa nell'ottenere le risorse. Uno dei problemi che hanno inciso maggiormente nella fase di sviluppo è consistito nel trovare gli errori quando le dinamiche di concorrenza non rispettavano le attese: ad ogni modo, il modello di concorrenza di Ada ci ha permesso di utilizzare meccanismi che rispecchiassero i comportamenti che volevamo ottenere, agevolando di molto la risoluzione dei problemi.\\

Diversamente, la distribuzione ha messo il luce le difficoltà nel coordinare entità remote che molto spesso sono regolate dall'interazione con l'utente. La scelta di utilizzare i WebSocket si è rivelata efficace e non ha presentato particolari problemi, salvo un maggiore accoppiamento all'interno del sistema. La parte che riguarda l'invio dei dati da parte di \emph{Core} e la loro successiva rappresentazione grafica si è rivelata invece più problematica, in quanto ha reso evidenti le necessità di adottare un filtro sugli eventi e un buffer di eventi regolato da un timer, sia per regolare il flusso di rete (riducendo quindi il rischio di congestioni) che per garantire una certa fluidità di visualizzazione della partita.\\

Infine, la scelta di usare Prolog per realizzare l'intelligenza artificiale dei giocatori ha avuto due effetti. Se da un lato ha permesso di esplorare un linguaggio appositamente progettato per questo tipo di compito e di avere quindi un'espressività maggiore da parte dei giocatori, dall'altro ha inficiato notevolmente la velocità di esecuzione del software. Quest'ultimo aspetto non era stato né previsto né tantomeno esplorato in fase di analisi, di conseguenza non c'è stato altro da fare che cercare di ridurlo quanto più possibile. Alla luce di ciò, l'uso di Prolog si è rivelato controproducente.\\