%----------------------------------------------------------------------------------------
%	ENTITA' COINVOLTE
%----------------------------------------------------------------------------------------

\section{Analisi}
\label{sec:analisi}

Il software che questo progetto mira a realizzare ha lo scopo di creare una simulazione di una partita di calcio. Nella fase di analisi è stata presa in considerazione una partita di calcio reale, al fine di definire tutte le funzionalità del software di simulazione, individuando così con precisione il problema da risolvere. Verrà quindi delineato un modello per la soluzione, mettendo in luce le entità coinvolte e il modo in cui esse interagiscono tra loro.

\subsection{Struttura del software}
\label{sec:struttura_del_software}

Il software di simulazione è costituito da tre diverse componenti, che prendono il nome (simbolico) di \textit{Core}, \textit{Field} e \textit{Manager}.

\paragraph{\textit{Core}} \label{sec:struttura_core} Il \textit{Core} è il modulo centrale del software. Esso contiene tutta la logica di gioco e regola l'interazione tra le diverse entità che lo compongono. È inoltre responsabile di gestire la comunicazione con i moduli esterni, ovvero \textit{Field} e i due \textit{Manager}, in quanto non dispone di una propria interfaccia grafica.

\paragraph{\textit{Field}} \label{sec:struttura_field} Questa componente rappresenta l'interfaccia grafica della partita. Attraverso di essa l'utente può iniziare una nuova partita, mettere in pausa quella corrente oppure chiudere del tutto la simulazione; in quest'ultimo caso, anche le componenti \textit{Core} e le due istanze di \textit{Manager} vengono terminate. Inoltre vi è una rappresentazione grafica molto semplice del campo di gioco, della panchina, dei giocatori e della palla. Viene inoltre riportato il tempo trascorso dall'inizio della partita, unitamente al punteggio corrente delle due squadre. Infine, è presente un registro di eventi che permette di ripercorrere testualmente gli eventi salienti della partita.

\paragraph{\textit{Manager}} \label{sec:struttura_manager} L'ultima componente è il \textit{Manager}, che permette di controllare una data squadra decidendo eventuali cambi di formazione e sostituzioni da effettuare. È possibile sostituire un giocatore alla volta, fino ad un massimo di tre, mentre non c'è limite ai cambi di formazione che è possibile fare. Le modifiche apportate dall'allenatore vengono applicate alla squadra in occasione della prima interruzione di partita dovuta ad un evento di gioco (la messa in pausa della partita da parte dell'utente non conta).\\

All'avvio del software, prima dell'inizio della partita, viene data la possibilità di configurare la propria squadra e i propri giocatori tramite un'interfaccia grafica dedicata. In essa i giocatori sono ordinati per ruolo, in base alla formazione scelta, e ciascuno di loro ha una scheda dedicata in cui sono riportate alcune caratteristiche fisiche configurabili. L'interfaccia da la possibilità di generarle casualmente per il giocatore selezionato, di generarle casualmente per tutti i giocatori della squadra oppure di inserire manualmente un valore per ciascuna di esse. Questa scelta deve essere ponderata perché le statistiche non sono più modificabili a partita iniziata. Anche la formazione iniziale della squadra è configurabile e può essere scelta fra tre possibili formazioni standard utilizzate nel mondo del calcio. La partita inizia non appena entrambe le squadre sono state configurate.

\subsection{Funzionalità del software}
\label{sec:funzionalita_del_software}

In questa sezione vengono introdotte le funzionalità del software di simulazione, che verranno poi dettagliate nei capitoli successivi.

\subsubsection{La partita}
\label{sec:analisi_partita}

Il software di simulazione permette di giocare una o più partite, ciascuna divisa in due tempi, la cui durata è prefissata e non modificabile dall'utente. L'inizio del secondo tempo è dettato dalla scelta dell'utente, così da permettergli eventuali modifiche all'assetto delle squadre. Lo svolgimento di una partita segue le regole ufficiali di base del calcio, eccezione fatta per il fuorigioco, che non viene preso in considerazione. Vi sono due squadre, identificate dai colori rosso e blu. Ciascuna squadra è composta da 11 giocatori in campo e 7 riserve in panchina, sostituibili attraverso una delle interfacce grafiche offerte all'utente.\\

\subsubsection{I giocatori}
\label{sec:analisi_giocatori}

I giocatori sono gli attori principali all'interno della simulazione di una partita. Essi devono essere liberi di agire sul campo di gioco ed avere quindi a disposizione una serie di azioni possibili, al fine di portare la propria squadra alla vittoria. A ciascun giocatore sono assegnate alcune caratteristiche di gioco che ne determinano la ``bravura'', intesa come la buona riuscita o meno delle azioni che vuole effettuare: tanto più alto il valore, tanto più ``bravo'' il giocatore in quel particolare aspetto di gioco. Si consideri come esempio un difensore che decide di rubare palla all'attaccante della squadra avversaria: la buona riuscita dell'intervento sarà quindi determinata dalla bravura (un valore alto) nel contrasto del primo e dalla capacità di scartare del secondo. Tali carattestiche sono riassumibili come segue e possono essere configurate prima dell'inizio della partita:

\begin{itemize}
	\item attacco
	\item difesa
	\item parata (valido solo per il portiere)
	\item velocità
	\item precisione
	\item potenza
	\item contrasto
\end{itemize}

Ogni giocatore in campo deve essere in grado di decidere autonomamente, in base alla situazione di gioco ed alle direttive dell'allenatore, le azioni da effettuare. Queste  si possono suddividere in tre gruppi: movimento nel campo, in ogni direzione, per raggiungere la posizione desiderata; interagire con la palla, quindi prenderla, spostarla con sé, passarla e tirarla in porta; infine, operazioni che comprendono un altro giocatore, per contrastare o scartare un avversario.\\

\subsubsection{L'arbitro}
\label{sec:analisi_arbitro}

Il corretto svolgimento della partita viene garantito dalla presenza di un arbitro, che provvede ad interrompere il gioco non appena si verifica un'infrazione (ad esempio un fallo) e a farlo ripartire non appena le condizioni lo permettono. Inoltre, l'arbitro ha il compito di sancire l'inizio e la fine di ciascun tempo, facendo opportunamente entrare in campo ed uscire in panchina tutti i giocatori. Infine, egli monitora la posizione della palla per fermare il gioco quando esce o se finisce in porta, aggiornando così il risultato della partita e riportando le squadre al centro del campo.\\

\subsubsection{Il campo}
\label{sec:analisi_campo}

Il campo da gioco deve essere suddiviso in celle, così da permettere una migliore gestione del movimento dei giocatori ed evitare che due di essi si trovino nella stessa posizione contemporaneamente. Le dimensioni del campo contano 51 celle sul lato lungo e 33 celle sul lato corto; volendo rispettare le proporzioni dettate dalle regole ufficiali del gioco, ogni cella corrisponde a 2 metri quadrati. Le panchine di entrambe le squadre sono adiacenti al campo, dove sono posizionati tutti i giocatori o solamente quelli di riserva a seconda del caso in cui ci sia una partita in corso o meno. È possibile assistere allo svolgimento della partita attraverso l'interfaccia grafica del campo di gioco, nella quale vengono visualizzate tutte le informazioni più importanti relativi alla partita (Sezione~\ref{sec:struttura_field}). Lo stato della partita può inoltre essere modificato, mettendola in pausa e facendola riprendere in un secondo momento, oppure terminarla prima della fine del tempo regolamentare. A partita terminata, è possibile iniziarne una nuova.\\

\subsubsection{Allenatori e squadra}
\label{sec:analisi_allenatori}

Gli allenatori hanno il compito di gestire le squadre ed i relativi giocatori in campo. Una decisione presa dall'allenatore si ripercuote sul gioco, andando a modificare le posizioni in campo dei giocatori ed il loro atteggiamento in fase offensiva e difensiva. Inoltre, entro i limiti imposti dalle regole di gioco, può effettuare alcune sostituzioni tra un giocatore in campo ed uno in panchina. È l'utente ad agire come allenatore, ogni qual volta lo ritiene opportuno, per effettuare le operazioni appena citate. In questo caso si necessita di una visione generale delle statistiche di ogni giocatore per rendere più facile prendere decisioni su quali sostituzioni attuare.

\subsection{Il modello}
\label{sec:entita_coinvolte}

Una volta definite la struttura generale e tutte le funzionalità del software, è possibile procedere ad identificare le entità e la loro tipologia. Per ciascuna di esse verrà quindi stabilito se rappresenta un'entità attiva, un'entità reattiva oppure una risorsa protetta, ed il loro ruolo all'interno dello scenario in cui si svolge la partita. Si procederà quindi a definire le interazioni che legano le suddette entità l'una con l'altra, così da poter derivare un modello per il sistema.\\

Per entità attive si considerano componenti dello scenario che necessitano di un proprio flusso di controllo, eseguendo ciclicamente le proprie azioni in modo autonomo. Le entità reattive non possiedono un proprio flusso di esecuzione, ma sono in attesa su canali opportunamente esposti per reagire alle richieste da parte di altre entità. Infine, la risorsa protetta differisce dall'entità reattiva in quanto risulta meno complessa, ma come essa mette a disposizione meccanismi per garantire mutua esclusione ed accodamento condizionale. La scelta di utilizzo tra le ultime due categorie è tipicamente dettata dalla mole di lavoro che deve essere svolta in risposta ad una richiesta da parte di un'altra entità.

\subsubsection{I giocatori}
\label{sec:entita_coinvolte_giocatori}

I giocatori vengono identificati come entità attive, quindi dotate di un proprio flusso di controllo che interagisce con le altre entità in gioco. Essi consisteranno in un numero di task pari al quantità di giocatori in campo, e ripeteranno ciclicamente il proprio corpo di esecuzione fino al termine della partita. Le decisioni che dovranno prendere saranno basate sullo stato circostante e sull'andamento dell partita. Nella restante parte di questo documento viene fatto riferimento al concetto di turno, inteso come singolo ciclo di esecuzione che viene ripetuto fino alla fine della partita. Tale turno si divide inoltre in più fasi, ma questo aspetto verrà ripreso in seguito nei prossimi capitoli.

%\subsubsection{Entità reattiva: l'arbitro}
%\label{sec:entita_coinvolte_arbitro}
%
%All'interno delle dinamiche del gioco l'arbitro reagisce alle azioni effettuate dalle altre entità, interrompendo o meno il gioco a seconda della circostanza e permett%endo, %se il momento è opportuno, ad entità esterne alla partita di interagire con le squadre in campo. Il suo scopo è quello di cambiare lo stato del gioco in modo tale da %influenzare il comportamento dei giocatori in campo, per esempio assegnando un fallo oppure una rimessa: così facendo, viene forzato il posizionamento di ognuno di loro ad %una determinata distanza dalla palla.

%\subsubsection{Risorsa protetta: la palla}
%\label{sec:entita_coinvolte_palla}
%
%La palla è costituita semplicemente da una coppia di coordinate, che rappresenta la posizione del campo (in termini di celle) in cui essa si trova. Al contrario dei giocatori,%che non possono occupare la medesima posizione, essa può trovarsi in una cella libera oppure in una occupata da un giocatore, che probabilmente la controlla. Subisce %spostamenti scaturiti da azioni dei giocatori o a causa di un tiro/passaggio. 

%\subsubsection{Entità attiva: agente di movimento}
%\label{sec:entita_coinvolte_agente}
%
%Durante lo svolgimento del gioco i giocatori interagiscono con la palla, il comportamento che ci si aspetta, salvo interventi da parte di avversa, è quello del gioco reale: %controllo, spostamento e tiro/passaggio. Dopo quest'ultima operazione la palla si sposta verso la direzione decisa con una velocità pari alla potenza impressa al pallone. Dato %che essa è una risorsa protetta, quindi non in grado di effettuare azioni complesse, è necessario introdurre una componente che all'occorrenza esegua spostando la palla da una %cella a quella adiacente fino a raggiungere una determinata posizione e lo faccia una certa velocità, sempre salvo interventi di altri giocatori. Tale entità è l'agente di %movimento, e la spiegazione del suo funzionamento viene rimandata alla parte del documento che parla dell'architettura e delle scelte fatte a livello di concorrenza.

%\subsubsection{Gli allenatori}
%\label{sec:entita_coinvolte_allenatori}
%
%Gli allenatori non sono un entità attiva, dato che non prendono decisioni automamente non abbiamo bisogno che essi abbiano un proprio flusso di controllo, bensì devono %reagire a segnali che arrivano dall'esterno. Il loro compito è quello di intermediare tra l'utente e il campo di gioco, influenzando le tattiche assunte dalle squadre e le %formazioni in campo. Successivamente questo aspetto sarà più chiaro quando verrà dato una visione generale del modello che comprenderà anche componenti di distribuzione. 

\subsubsection{Lo stato}
\label{sec:entita_coinvolte_stato}

In fase di analisi è stato messo in evidenza il fatto che un giocatore ha bisogno di apprendere lo stato di gioco per poter decidere la sua mossa successiva. Questo porta alla luce la necessità di avere un luogo unico in cui detenere lo stato di gioco, consultabile dai giocatori ogni qual volta ne abbiano bisogno. Tale stato consiste principalmente nelle posizioni degli altri giocatori in campo e l'andamento del gioco, come per esempio il fatto che la sua squadra sia in fase offensiva o difensiva.\\

Quello che un giocatore è chiamato a fare è quindi, a grandi linee, ottenere lo stato di gioco in un determinato istante, calcolare la propria mossa ed andare ad aggiornare lo stato in base a quest'ultima decisione.\\

Queste considerazioni evidenziano la necessità di aggiungere una nuova entità che governi lo stato, che ne garantisca l'integrità nel proseguire del gioco e garantisca ai giocatori di poter leggere lo stato e modificarlo secondo le mosse scelte: il controllore.
