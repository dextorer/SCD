%----------------------------------------------------------------------------------------
%	COMPILAZIONE ED ESECUZIONE
%----------------------------------------------------------------------------------------

\section*{Compilazione ed esecuzione}
\addcontentsline{toc}{section}{\protect\numberline{}Compilazione ed esecuzione}%
\label{sec:compilazione_esecuzione}
Per la compilazione e l'esecuzione del progetto sono necessari i seguenti software:
\begin{itemize}
\item GNAT_GPL, ambiente di sviluppo;
\item AWS, web server scritto in ADA;
\item XMLAda, librerie per il parsing XML;
\item GNATColl, libreria ADA usata per la gestione di dati in formato JSON; 
\item Java utilizzato nello sviluppo delle interfacce grafiche;
\item Prolog, in particolare SWI-Prolog, utilizzato nello sviluppo dell'intelligenza artificiale dei giocatori;
\item Ada Util, un insieme di packages per Ada che forniscono funzionalit\`{a} aggiuntive.
\end{itemize}
\noindent Tutti questi software sono gratuiti e liberamente scaricabili. In particolare GNAT_GPL, AWS, XMLAda e GNATColl sono sviluppati e supportati da Ada-Core e sono disponibili al sito http://libre.adacore.com sotto licenza GPL. Bisogna inoltre fare attenzione ad avere una versione di Java non inferiore a Java 6. Come detto in precedenza l'implementazione di Prolog utilizzata \`{e} SWI_prolog, scaricabile gratuitamente dal sito http://www.swi-prolog.org/ in licenza LGPL. Infine Ada Util \`{e} reperibile al sito http://code.google.com/p/ada-util/ in licenza Apache 2.0. 

Il progetto \`{e} stato testato nelle distribuzioni Linux Ubuntu e Mint e in Mac OS X.

\noindent Di seguito vengono proposti i passaggi da eseguire per avere un ambiente di sviluppo pronto, in grado di compilare il progetto. La distribuzione di riferimento \`{e} Mint desktop 15 i386. Per ulteriori dettagli si pu\`{o} far riferimento ai file README o INSTALL presenti negli archivi dei file scaricati.
\begin{itemize}
\item Scompattare [GNAT FILE]
	\begin{itemize}
	\item[] installare l'ambiente con il comando ./doinstall
	\item[] esportare le variabili d'ambiente in questo modo
	\item[] export PATH=/usr/gnat/bin:\$PATH
	\item[] export GPR_PROJECT_PATH=/usr/gnat/lib/gnat
	\item[] export ADA_PROJECT_PATH=/usr/gnat/lib/gnat
	\end{itemize}
\item Scompattare [XML ADA FILE] e lanciare i comandi
	\begin{itemize}
	\item[] ./configure --prefix=/usr/gnat
	\item[] make all
	\item[] make install (eseguire da root)
	\end{itemize}
\item Scompattare [AWS FILE] e lanciare i seguenti comandi
	\begin{itemize}
	\item[] make setup
	\item[] make build
	\item[] make install (eseguire da root)
	\end{itemize}
\item Scompattare [GNATCOLL FILE] e lanciare i seguenti comandi
	\begin{itemize}
	\item[] ./configure
	\item[] make
	\item[] make prefix=/usr/gnat install (eseguire da root)
	\end{itemize}
\item [AUNIT]
\item [ADA UTIL]
\item L'installazione di SWI-Prolog \`{e} abbastanza semplice, in quanto gli sviluppatori mettono a disposizione un repository comodamente aggiungibile al package manager di Ubuntu e di tutte le distribuzione basate su Ubuntu 
	\begin{itemize}
	\item[] sudo apt-add-repository ppa:swi-prolog/stable
	\item[] sudo apt-get update
	\item[] sudo apt-get install swi-prolog
	\end{itemize}
\end{itemize}

\noindent A questo punto l'ambiente di sviluppo \`{e} pronto. Per compilare ed eseguire il progetto bisogna aprire un terminale, spostarsi nella cartella del progetto ed eseguire il comando\\
sh run_all.sh\\
Con tale comando viene lanciato lo script che si occupa di compilare tutto il progetto e di avviarlo. Al termine della compilazione compariranno le GUI di configurazione della squadra dalle quali sar\`{a} poi possibile iniziare la partita.